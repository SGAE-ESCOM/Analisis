\section{Reglas de Negocio}
\begin{BussinesRule}{BR1}{}
    \BRitem[Tipo:]
    \BRitem[Clase:]
    \BRitem[Nivel:]
    \BRitem[Descripción:]Para la creación del formulario  al menos debe existir un requisito.
    
\end{BussinesRule}
%----------------------------------BR --------------------------------------%
\begin{BussinesRule}{BR2}{}
    \BRitem[Tipo:]
    \BRitem[Clase:]
    \BRitem[Nivel:]
    \BRitem[Descripción:]Un requisito deberá conformarse de nombre y tipo de dato.
    
\end{BussinesRule}
%----------------------------------BR --------------------------------------%
\begin{BussinesRule}{BR3}{}
    \BRitem[Tipo:]
    \BRitem[Clase:]
    \BRitem[Nivel:]
    \BRitem[Descripción:]El nombre del tipo de dato es alfanumérico.
    
\end{BussinesRule}
%----------------------------------BR --------------------------------------%
\begin{BussinesRule}{BR4}{}
    \BRitem[Tipo:]
    \BRitem[Clase:]
    \BRitem[Nivel:]
    \BRitem[Descripción:]Los tipos de datos que se podrán introducir serán campo, archivo, de selección y fecha. 
    
\end{BussinesRule}
%----------------------------------BR --------------------------------------%
\begin{BussinesRule}{BR5}{}
    \BRitem[Tipo:]
    \BRitem[Clase:]
    \BRitem[Nivel:]
    \BRitem[Descripción:]El tipo de dato campo solo podrá ser de texto y número
    
\end{BussinesRule}
%----------------------------------BR --------------------------------------%
\begin{BussinesRule}{BR6}{}
    \BRitem[Tipo:]
    \BRitem[Clase:]
    \BRitem[Nivel:]
    \BRitem[Descripción:]El tipo de dato archivo solo podrá ser una imagen con extensión png y jpg, o  pdf.
    
\end{BussinesRule}
%----------------------------------BR --------------------------------------%
\begin{BussinesRule}{BR7}{}
    \BRitem[Tipo:]
    \BRitem[Clase:]
    \BRitem[Nivel:]
    \BRitem[Descripción:]El tipo de dato de selección podrá ser con opción única o múltiple.
    
\end{BussinesRule}
%----------------------------------BR --------------------------------------%
\begin{BussinesRule}{BR8}{}
    \BRitem[Tipo:]
    \BRitem[Clase:]
    \BRitem[Nivel:]
    \BRitem[Descripción:]En el tipo de dato de fecha se deberá seleccionar el rango de tiempo.
    
\end{BussinesRule}
%----------------------------------BR --------------------------------------%
\begin{BussinesRule}{BR9}{}
    \BRitem[Tipo:]
    \BRitem[Clase:]
    \BRitem[Nivel:]
    \BRitem[Descripción:]Creado el requisito, toda las caracteristicas  se podrán editar a excepción del tipo de dato.
    
\end{BussinesRule}
%----------------------------------BR --------------------------------------%
\begin{BussinesRule}{BR10}{}
    \BRitem[Tipo:]
    \BRitem[Clase:]
    \BRitem[Nivel:]
    \BRitem[Descripción:]Un requisito tendrá la opción de ser obligatorio o no.
    
\end{BussinesRule}
%----------------------------------BR --------------------------------------%
\begin{BussinesRule}{BR11}{}
    \BRitem[Tipo:]
    \BRitem[Clase:]
    \BRitem[Nivel:]
    \BRitem[Descripción:]En la vista previa del formulario,  se podrá verificar mediante el botón finalizar. ????
    
\end{BussinesRule}
%----------------------------------BR --------------------------------------%
\begin{BussinesRule}{BR12}{}
    \BRitem[Tipo:]
    \BRitem[Clase:]
    \BRitem[Nivel:]
    \BRitem[Descripción:]En caso de recibir una validación aprobatoria, se deberá esperar a la siguiente etapa del proceso.
    
\end{BussinesRule}
%----------------------------------BR --------------------------------------%
\begin{BussinesRule}{BR13}{}
    \BRitem[Tipo:]
    \BRitem[Clase:]
    \BRitem[Nivel:]
    \BRitem[Descripción:]En caso de recibir una validación desaprobada, solo se habilitara el requisito a modificar y la opción de volver a enviar formulario.
    
\end{BussinesRule}
%----------------------------------BR --------------------------------------%
\begin{BussinesRule}{BR14}{}
    \BRitem[Tipo:]
    \BRitem[Clase:]
    \BRitem[Nivel:]
    \BRitem[Descripción:]La revisión para validación de un alumno solo se hará por un adminitsrador  de la institución a la vez.
    
\end{BussinesRule}
%----------------------------------BR --------------------------------------%
\begin{BussinesRule}{BR15}{}
    \BRitem[Tipo:]
    \BRitem[Clase:]
    \BRitem[Nivel:]
    \BRitem[Descripción:]La pantalla de visualizar alumnos contendrá una tabla con nombre(s) del alumno, status de validación y acciones.
    
\end{BussinesRule}
%----------------------------------BR --------------------------------------%
\begin{BussinesRule}{BR16}{}
    \BRitem[Tipo:]
    \BRitem[Clase:]
    \BRitem[Nivel:]
    \BRitem[Descripción:]En caso de que todos los requisitos estén completos y validados, se enviara la notificación a alumno de esto, al igual que si no están validados, se mandara la notificación de que requisito(s) corregir, anexando el comentario del error. 
    
\end{BussinesRule}
%----------------------------------BR --------------------------------------%
\begin{BussinesRule}{BR17}{}
    \BRitem[Tipo:]
    \BRitem[Clase:]
    \BRitem[Nivel:]
    \BRitem[Descripción:]
    
\end{BussinesRule}
%----------------------------------BR --------------------------------------%
\begin{BussinesRule}{BR18}{}
    \BRitem[Tipo:]
    \BRitem[Clase:]
    \BRitem[Nivel:]
    \BRitem[Descripción:]
    
\end{BussinesRule}
%----------------------------------BR --------------------------------------%
\begin{BussinesRule}{BR19}{Máquina de Estados de una Tarea.}
    \BRitem[Tipo:] Flujo.
    \BRitem[Clase:] Habilitadora.
    \BRitem[Nivel:] Control.
    \BRitem[Descripción:] Según el estado de la Tarea son los permisos de quien puede modificar la información relacionada con esta.
     Tener un mayor control sobre la información.
    \BRitem[Ejemplo Positivo:] La Tarea de registrar Mapa Curricular se aprobó después de que el analista y el jefe la revisaran.
    \BRitem[Ejemplo Negativo:] La Tarea de registrar Unidad de Aprendizaje se empezó a revisar antes de que terminara su registro.
\end{BussinesRule}
%----------------------------------BR --------------------------------------%
\begin{BussinesRule}{BR20}{Aprobación de Tareas.}
    \BRitem[Tipo:] Flujo.
    \BRitem[Clase:] Habilitadora.
    \BRitem[Nivel:] Control.
    \BRitem[Descripción:] Una Tarea es aprobada si al terminar de revisar no contiene comentarios.
     Poder saber cuando una tarea a finalizado.
    \BRitem[Ejemplo Positivo:] El Jefe de Innovación Educativa no agrego comentarios por lo que se aprueba.
    \BRitem[Ejemplo Negativo:] El Jefe de Innovación Educativa agrega comentario y aun así se aprueba el documento.
\end{BussinesRule}
%----------------------------------BR --------------------------------------%
\begin{BussinesRule}{BR21}{Aprobación de Tareas Seccionada.}
    \BRitem[Tipo:] Condición.
    \BRitem[Clase:] Habilitadora.
    \BRitem[Nivel:] Control.
    \BRitem[Descripción:]
    \BRitem[Sentencia:] Una tarea puede ser aprobada parcialmente y las secciones aprobadas no pueden ser modificadas.
     Evitar modificaciones en una sección ya aprobada.
    \BRitem[Ejemplo Positivo:] Una sección ya ha sido aprobada y no se hacen modificaciones.
    \BRitem[Ejemplo Negativo:] Una sección es aprobada y fue modificada.
\end{BussinesRule}
%----------------------------------BR --------------------------------------%
\begin{BussinesRule}{BR22}{Rechazo de Tareas.}
    \BRitem[Tipo:] Flujo.
    \BRitem[Clase:] Habilitadora.
    \BRitem[Nivel:] Control.
    \BRitem[Descripción:] Una Tarea es rechazada si al terminar de revisar contiene comentarios.
     Poder saber cuando una tarea debe volver al estado de Registro.
    \BRitem[Ejemplo Positivo:] El Jefe de Innovación Educativa agrego comentarios por lo que se rechaza.
    \BRitem[Ejemplo Negativo:] Ni el Jefe de Innovación Educativa ni el Analista agregaron comentarios y aun así se aprobó.
\end{BussinesRule}
%----------------------------------BR --------------------------------------%
\begin{BussinesRule}{BR23}{Tiempos de entregas.}
    \BRitem[Tipo:] Integridad.
    \BRitem[Clase:] Cronometrado.
    \BRitem[Nivel:] Control.
    \BRitem[Descripción:] Una tarea debe entregarse en la fecha indicada por el jefe de desarrollo e innovación curricular.
     Se tiene un control en las entregas de tareas.
    \BRitem[Ejemplo Positivo:] Una propuesta de unidad de aprendizaje puede ser revisada por un analista solo en el tiempo establecido.
    \BRitem[Ejemplo Negativo:] Una propuesta de unidad de aprendizaje puede ser revisada por un analista en cualquier momento.
\end{BussinesRule}
%----------------------------------BR --------------------------------------%
\begin{BussinesRule}{BR24}{Todos los datos solicitados son obligatorios.}
    \BRitem[Tipo:] Regla de Operación.
    \BRitem[Clase:] Habilitadora.
    \BRitem[Nivel:] Control.
    \BRitem[Descripción:] Los campos solicitados, no se pueden ser dejados en blanco.
    \BRitem[Sentencia:]
    \BRitem[Motivación: ]Que la base de datos esté siempre en un estado consistente.
    \BRitem[Ejemplo Positivo:] El usuario ingresa todos los datos solicitados y prosigue con su operación.
    \BRitem[Ejemplo Negativo: ]El usuario deja campos vacios y el sistema le permite continuar.
\end{BussinesRule}
%----------------------------------BR --------------------------------------%
\begin{BussinesRule}{BR25}{El correo electrónico del empleado es único.}
    \BRitem[Tipo: ]Relación.
    \BRitem[Clase: ]Habilitadora.
    \BRitem[Nivel: ]Control.
    \BRitem[Descripción:] Cada una de las cuentas de los empleados registrados en el sistema cuentan con un correo electrónico único.
    \BRitem[Motivación: ]Poder identificar a los usuarios.
    \BRitem[Ejemplo Positivo:] El Empleado Juan Perez tiene una correo electrónico juanp@ipn.mx  y el empleado Armando López Doriga tiene un correo electrónico armlop@ipn.mx.
    \BRitem[Ejemplo Negativo:] El Empleado Juan Perez tiene una correo electrónico juanp@ipn.mx  y el empleado Armando López Doriga tiene un correo electrónico juanp@ipn.mx.
\end{BussinesRule}
%----------------------------------BR --------------------------------------%

\begin{BussinesRule}{BR26}{Debe existir al menos un criterio de evaluación para una Unidad de Aprendizaje.}
    \BRitem[Tipo:] Regla de integridad estructural.
    \BRitem[Clase:] Habilitadora.
    \BRitem[Nivel:] Control.
    \BRitem[Descripción:] Para cada Unidad de Aprendizaje debe existir al menos un criterio de evaluación cuyo porcentaje asociado sería de 100\%, sin embargo, puede tener tantos criterios como el docente decida.
     Que el docente que imparta dicha Unidad de Aprendizaje tenga al menos un referente para evaluar a sus alumnos.
    \BRitem[Ejemplo positivo:] La Unidad de Aprendizaje ``Ingeniería de Sofware'' tiene como único criterio de evaluación el desarrollo de un proyecto cuyo porcentaje equivale al 100\% de la calificación del alumno.
\end{BussinesRule}
%----------------------------------BR --------------------------------------%
\begin{BussinesRule}{BR27}{La suma de los porcentajes de cada evaluación debe ser igual a 100\%.}
    \BRitem[Tipo:] Regla de operación.
    \BRitem[Clase:] Habilitadora.
    \BRitem[Nivel:] Control.
    \BRitem[Descripción:] La suma total de los porcentajes de cada evaluación registrada por el docente, debe ser exactamente igual al 100\%.
     Que exista coherencia al momento de evaluar y que el alumno pueda identificar de dónde proviene su calificación.
    \BRitem[Ejemplo positivo:] Las evaluaciones registradas para la Unidad de Aprendizaje ``Ingeniería de Software'' son:
        \begin{itemize}
            \item 20\% Examen oral.
            \item 20\% Tareas.
            \item 60\% Proyecto final.
        \end{itemize}
        En total, los porcentajes suman el 100\% de la calificación del alumno.
\end{BussinesRule}
%----------------------------------BR --------------------------------------%
   \begin{BussinesRule}{BR28}{Dirección web del sistema.}
     \BRitem[Tipo:] Flujo.
     \BRitem[Clase:] Habilitadora.
     \BRitem[Nivel:] Control.
     \BRitem[Descripción:] La dirección web a la cual tendrán que ingresar los usuarios en su navegador para acceder al sistema es la siguiente: \url{https://sgae-escom.firebaseapp.com/#/}
      Informar a los usuarios cómo acceder al sistema una vez puesto en línea.
  \end{BussinesRule}
 %----------------------------------BR --------------------------------------%
   \begin{BussinesRule}{BR29}{Verificación de formularios al momento.}
     \BRitem[Tipo:]  Estímulo y respuesta.
     \BRitem[Clase:] Habilitadora.
     \BRitem[Nivel:] Control.
     \BRitem[Descripción:] El sistema realiza las verificaciones conforme al Modelo de Datos mientras el Usuario ingresa los datos.
      Evitar inconsistencias en los datos.
  \end{BussinesRule}
  %----------------------------------BR --------------------------------------%
\begin{BussinesRule}{BR30}{Todos los campos marcados con (*) son obligatorios.}
    \BRitem[Tipo:] Regla de Operación.
    \BRitem[Clase:] Habilitadora.
    \BRitem[Nivel:] Control.
    \BRitem[Descripción:] Los campos marcados con un (*), no se pueden ser dejados en blanco.
    \BRitem[Sentencia:]
    \BRitem[Ejemplo Positivo:] El usuario ingresa los campos obligatorios y prosigue con su operación.
    \BRitem[Ejemplo Negativo: ]El usuario deja campos obligatorios vacíos y el sistema le permite continuar.
\end{BussinesRule}
%----------------------------------BR --------------------------------------%
\begin{BussinesRule}{BR41}{Solo puede haber un plan de estudios en estado de creación, rediseño y aprobado.}
    \BRitem[Tipo:] Flujo.
    \BRitem[Clase:] Habilitadora.
    \BRitem[Nivel:] Control.
    \BRitem[Descripción:] El usuario solo puede existir un plan de estudios en estado de creación, rediseño y aprobado.
     Evitar la creación de un Plan de Estudio sin la aprobación.
\end{BussinesRule}
\pagebreak