% MODULO DE DOCUMENTACIÓN (MD) REGISTRAR REQUISISTOS DE DOCUMENTACIÓN
\begin{UseCase}{MD-CU1}{Registrar Requisitos de Documentación//Información General}{El actor podrá registrar los Requisitos correspondiente a una Convocatoria.} 
    \UCitem{Versión}{\color{Gray}1.1}
    \UCitem{Autor}{\color{Gray}López Rivera Aiko Dallane.}
    \UCitem{Supervisa}{\color{Gray}Cervantes Moreno Christian Andrés.}
    \UCitem{Actor}{\hyperlink{Docente}{Administrador de la Institución Educativa.}}
    \UCitem{Propósito}{Servir como marco de referencia para el registro de los demás atributos de la Convocatoria.}
    \UCitem{Entradas}{Las entradas para el registro de los requisitos serán:
      \begin{itemize}
          \item Nombre de requisito
          \item Tipo de campo
          \item Subtipo
          \item Campo requerido o no. 
          \item Mínimo y/o máximo de caracteres 
          \item Letras mayúsculas
          \item Letras minúsculas
          \item Números
          \item Espacios
          \item Expresión regular
          \item Descripción
          \item Opciones
          \item Fecha máxima
          \item Fecha mínima 
      \end{itemize}
    }
    \UCitem{Origen}{Teclado y Mouse.}
    \UCitem{Salidas}{
    	\begin{itemize}
    		\item Formulario
        	\item \MSGref{MSG1}{Llena todos los campos requeridos.}
        	\item \MSGref{MSG2}{Se agrego correctamente}
	        \item \MSGref{MSG3}{¿Desea eliminar "requisito"?, No se podrá revertir está acción.}
    		\item \MSGref{MSG4}{Eliminado, El elemento se ha eliminado}
          	\item  \MSGref{MSG5}{Registro finalizado exitosamente.}
          	\item  \MSGref{MSG6}{Mensaje de prueba. Se ha guardado la información.}
          	\item  \MSGref{MSG7}{Mensaje de prueba. El formulario es válido.}
          	\item  \MSGref{MSG8}{Mensaje de prueba. El formulario no es válido.}
          	\item  \MSGref{MSG9}{El requisito se actualizo exitosamente.}
          	\item  \MSGref{MSG10}{Por favor escribe un nombre.}
            \item  \MSGref{MSG11}{Por favor elige una opción.}
            \item  \MSGref{MSG12}{Por favor elige un tipo.}
            \item  \MSGref{MSG13}{Este campo es requerido.}
          	
     	\end{itemize}
    }
    \UCitem{Destino}{Pantalla.}
    \UCitem{Precondiciones}{El actor debe definir las etapas y tiempos de cada una.}
    \UCitem{Postcondiciones}{El Formulario de requisitos queda registrado en el Sistema.}
    \UCitem{Errores}{}
     \UCitem{Puntos de extensión}{
     \begin{itemize}
        \item\UCref{MD-CU3}: Editar Requisitos
    \end{itemize}    
    }
    \UCitem{Estado}{Revisión.}
    \UCitem{Observaciones}{}
\end{UseCase}

%--------------------------- CU TRAYECTORIA PRINCIPAL -------------------------
\begin{UCtrayectoria}{Principal}

\UCpaso[\UCactor] Presiona el botón \IUbutton{Administrar} de la interfaz de usuario \IUref{DOC}{Documentación}.

\UCpaso Carga la información del Formulario para Creación de Requisito .

\UCpaso Muestra la interfaz de usuario \IUref{ARE}{Administrar requisitos}.

\UCpaso[\UCactor] Define un requisito con sus respectivas características, según la regla de negocio\BRref{BR1}{Todos los campos marcados con (*) son obligatorios}.

\UCpaso[\UCactor] Termina la operación presionando el botón agregar.[Trayectoria A][Trayectoria B].

\UCpaso Muestra una tabla con los requisitos diseñados en el paso anterior.[Trayectoria C].

\UCpaso Muestra una tabla de prueba de formulario para aspirante.
%Preguntar a rabadán.
\UCpaso[\UCactor] Ingresa un ejemplo de los requisitos diseñados en el paso 6, el sistema verifica conforme al modelo de datos, la \BRref{BR2}{Verificación de formularios al momento} y la \BRref{BR1}{Todos los campos marcados con (*) son obligatorios}. [Trayectoria D] [Trayectoria E]

\UCpaso[\UCactor] Termina la operación presionando el botón \IUbutton{Finalizar}. [Trayectoria F]
\UCpaso El sistema muestra el mensaje \MSGref{MSG7}{Mensaje de prueba. El formulario es válido.}


\end{UCtrayectoria}



%------------------------ CU TRAYECTORIA ALTERNARTIVA A -------------------------
\begin{UCtrayectoriaA}{A}{Uno o más campos obligatorios no fueron contestados.}
\UCpaso Detecta uno o más campos sin contestar.
\UCpaso Muestra el mensaje \MSGref{MSG1}{Llena todos los campos requeridos} debajo de los campos que no fueron contestados.

\UCpaso Continua en el paso 4 de la trayectoria principal del \UCref{SP1-CU1}
\end{UCtrayectoriaA}

%------------------------ CU TRAYECTORIA ALTERNARTIVA E -------------------------
\begin{UCtrayectoriaA}{E}{El sistema detecta caracteres no válidos conforme al diccionario de datos.}
	\UCpaso Muestra el mensaje \MSGref{MSG35}{Escribe información válida} debajo del campo que incumplió según el diccionario de datos.
	\UCpaso Continúa en el paso 4 de la trayectoria principal del \UCref{SP1-CU1}.
\end{UCtrayectoriaA}

%------------------------ CU TRAYECTORIA ALTERNARTIVA F -------------------------
\begin{UCtrayectoriaA}{F}{Ocurre un error al momento de persistir los datos.}
	\UCpaso Muestra el mensaje \MSGref{MSG25}{Servicios no disponibles}.
	
	\UCpaso[\UCactor] Cierra el mensaje presionando el botón \IUbutton{Aceptar}.
	
	\UCpaso Muestra la interfaz de usuario \IUref{RPS}{Registrar Programa Sintético}.
\end{UCtrayectoriaA}

%------------------------ CU TRAYECTORIA ALTERNARTIVA G -------------------------
\begin{UCtrayectoriaA}{G}{El actor aún no desea finalizar el registro}
	\UCpaso[\UCactor] Presiona el botón \IUbutton{No}.
	
	\UCpaso Cierra el mensaje.
	
	\UCpaso Continúa en el paso 10 de la trayectoria principal del \UCref{SP1-CU1}.
\end{UCtrayectoriaA}