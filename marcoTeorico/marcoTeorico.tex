\chapter{Marco Teórico} %CAP\'ITULO 2
    
    El marco teórico, que se desarrolla a continuación, permite conocer los conceptos básicos necesarios para el entendimiento del proyecto. 
    \section{Admisión Escolar}
    La administración escolar es un campo amplio que abarca casi cualquier tema relacionado con el funcionamiento de una institución académica, desde la gestión de un programa preescolar hasta el desarrollo de programas de doctorado.
    \section{Sistema}
    Un sistema es una disposición de partes o elementos que juntos exhiben un comportamiento o significado que los constituyentes individuales no tienen.
    Los sistemas pueden ser físicos o conceptuales, o una combinación de ambos.
    \section{Sistema Web}

    \section{Convocatoria}
    \section{Inscripción}
    \section{FireBase}
    Firebase es una tecnología que le permite crear aplicaciones web sin programación del lado del servidor, haciendo que el desarrollo sea más rápido y fácil. Es compatible con clientes web, iOS, OS X y Android. Las aplicaciones que usan Firebase pueden usar y controlar datos sin pensar en cómo se almacenan y sincronizan los datos en diferentes instancias de la aplicación en tiempo real.

    Trabajar con Firebase desde la perspectiva de un desarrollador es un beneficio maravilloso, ya que son la tecnología central del desarrollo.
    
    Ventajas:
    
    Autenticación. La autenticación de Firebase incluye un sistema de autenticación de correo electrónico / contraseña incorporado. Es compatible con OAuth2 para Facebook, Google, Twitter y GitHub. Además, el estándar Firebase se integra directamente en la base de datos Firebase para que pueda usarlo para controlar el acceso a sus datos.
    Hospedaje Firebase viene con un servicio de alojamiento fácil de usar para todos sus archivos estáticos. Funciona desde un CDN global con HTTP / 2.
    La sincronización de datos en tiempo real en todos los clientes, ya sea Android, iOS o la Web, es muy útil. Con un código mínimo, puede notificar a los usuarios de los cuadros de chat, noticias en vivo, nuevas publicaciones o solicitudes de amistad, y más.
    El código para AJS es sencillo de cualquier manera. Desde la consulta de datos hasta la integración de los inicios de sesión de Twitter, Facebook y Google+, puede implementarlos muy rápidamente con algunas características interesantes.
    Con notificaciones de actualización automática, puede sincronizar ambos sistemas sin mensajes manuales, WebSockets, etc.
    Le permite considerar flujos de datos para crear aplicaciones más escalables.
    Algunos beneficios de usar Firebase
    
    \begin{itemize}
        \item Base de datos en tiempo real de Firebase
        \item Estándar de Firebase
        \item Almacenamiento Firebase
        \item Mensaje de Firebase Cloud
        \item Notificación de Firebase
        \item Configuración remota de Firebase
        \item Informe de bloqueo de Firebase
        \item Índice de aplicaciones de Firebase
        \item Firebase Analytics
        \item Firebase Test Lab para Android
    \end{itemize}
    
    \section{Node JS}
    
    Nodo. js es una plataforma basada en el tiempo de ejecución de JavaScript de Chrome para crear fácilmente aplicaciones de red rápidas y escalables. Nodo. js utiliza un modelo de E / S sin bloqueo controlado por eventos que lo hace liviano y eficiente, perfecto para aplicaciones en tiempo real de uso intensivo de datos que se ejecutan en dispositivos distribuidos.
    \section{Angular}
    AngularJS es un marco estructural para aplicaciones web dinámicas. Le permite usar HTML como su lenguaje de plantilla y le permite extender la sintaxis de HTML para expresar los componentes de su aplicación de manera clara y sucinta. El enlace de datos de AngularJS y la inyección de dependencia eliminan gran parte del código que de lo contrario tendría que escribir. Y todo sucede dentro del navegador, lo que lo convierte en un socio ideal con cualquier tecnología de servidor.
    AngularJS es lo que habría sido HTML, si hubiera sido diseñado para aplicaciones. HTML es un gran lenguaje declarativo para documentos estáticos. No contiene mucho en cuanto a la creación de aplicaciones, y como resultado, la creación de aplicaciones web es un ejercicio de ¿qué debo hacer para engañar al navegador para que haga lo que quiero?
    %\section{Scrum}
    %\section{Evaluación}    
    %\section{Simulación}
    %\section{Arquitectura}
    %\section{Escalabilidad}
    %\section{Usalabilidad}
    %\section{Métricas}
    %\section{Mantenible}
    %\section{Metodología}    
    %\section{Regla de Negocio}
    %\section{Caso de Uso}
    %\section{Diagrama de Secuencia}
    %Andrew hace esto
    \section{Diagrama de Proceso}
    Es la representación gráfica de un conjunto de actividades, acciones o toma de decisiones interrelacionadas, caracterizadas por inputs y outputs, orientadas a obtener un resultado específico como consecuencia del valor añadido aportado por cada una de las actividades que se llevan a cabo en las diferentes etapas de dicho proceso
    \section{Interfaz de Usuario}
    Las interfaces de usuario son todo aquel espacio gráfico y físico en donde los usuarios interactúan con el software
    \section{Diagrama de componentes}
    Los diagramas de componentes son esencialmente diagramas de clase que se centran en los componentes de un sistema que a menudo se utilizan para modelar la vista de implementación estática de un sistema.
    \section{Máquina de Estados}
    Una máquina de estados es un modelo de comportamiento. Consiste en un número finito de estados, según el estado actual y una entrada dada, la máquina realiza transiciones de estado y produce salidas.
    \section{Configurable}
    Configurable es que el comportamiento de determinadas funcionalidades, la realización de determinados cálculos o la aplicación de determinadas restricciones varíe en tiempo de ejecución, también es la posibilidad de poder actualizar manualmente. 
    \section{Discente}
     Persona que recibe un aprendizaje y unos conocimientos de otra persona (generalmente de un maestro).
    \section{Pruebas de Caja Negra}
    Es una técnica de pruebas de software en la cual la funcionalidad se verifica sin tomar en cuenta la estructura interna de código, detalles de implementación o escenarios de ejecución internos en el software.
    \section{Gestión}
    Hace la referencia a la administración de recursos, sea dentro de una institución estatal o privada, para alcanzar los objetivos propuestos por la misma.
    \section{Inyección de Código}
     Es el proceso de introducir a un programa ó sistema software una serie de instrucciones que no formaban parte de la composición original del  programa/sistema, provocando modificaciones en el funcionamiento original del programa/sistema, y en su rendimiento .
    \section{Integridad}%Seguridad
     Es la cualidad que posee un documento que no ha sido alterado y que además permite comprobar que no sea manipulado el documento original.
    \section{Autenticación}%S
    Es la comprobación de la identidad de una persona o de un objeto
    \section{Confidencialidad}%S
    Se conoce en derecho como una comunicación privilegiada, la cual se define como un intercambio de información entre dos personas en una relación entre el profesional y su cliente, en la cual la relación confidencial es expresamente reconocida por ley. 
    \section{Patrones de diseño}
    \section{MVVM}
    MVVM es una arquitectura desarrollada por Microsoft alrededor de 2004, cuando también se creó Windows Presentation Foundation.
    \section{Protocolo HTTPS}
    El Protocolo seguro de transferencia de hipertexto es un protocolo de aplicación basado en el protocolo HTTP, destinado a la transferencia segura de datos de hipertexto, es decir, es la versión segura de HTTP.
    
    