\section{Contexto de Trabajo.}
    La era digital ha cambiado drásticamente los procesos de admisión en instituciones y universidades alrededor del mundo. La innovación, las  tecnologías en la nube, móviles y digitales están acelerando la transformación educativa, mejorando las experiencias tanto de docentes como de aspirantes [1]. Los pasos del proceso de admisión escolar usualmente son: publicación de convocatoria, registro de aspirantes, aplicación de examen y publicación de resultados, sin embargo, no todas las instituciones se apegan al mismo proceso de admisión; por lo que se propone un sistema capaz de ser configurable entre las etapas de admisión y adaptable a las instituciones.
    
    Para las instituciones de educación superior en México, el proceso de admisión consiste en elegir dentro de un conjunto de postulantes a los futuros aspirantes mejor capacitados en los programas académicos de su preferencia. Para llevar a cabo este proceso, los aspirantes a una vacante requieren de conocimientos generales relacionados al programa académico de su preferencia, además, será necesario que sigan el orden de las etapas establecidas según la institución a la que aplican.