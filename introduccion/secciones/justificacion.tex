\section{Justificación}

    De acuerdo con estadísticas de los periódicos el Universal [7] y Excélsior [8], el número de aspirantes que se postulan para ingresar al IPN y la UNAM en 2018 sumó un total de  236 mil jóvenes. Tomando como parámetro el gran número de solicitantes, resulta común que exista error en alguna de las fases del proceso de admisión, errores que, en su mayoría, fueron por actividades que son realizadas de forma manual. 
    Cuando un aspirante hace la entrega de documentos con su respectiva institución, es necesario hacer largas filas y  pasar por una cantidad definida de filtros humanos para llegar a la conclusión del proceso de recepción de documentación, pero hay un problema palpable en este hecho: los seres humanos son susceptibles a cometer errores, por tanto, existe la posibilidad de que al momento de recopilar todos los expedientes, se extravíen documentos; provocando que la institución pierda prestigio por la pérdida de información así como el tiempo perdido en las aclaraciones correspondientes. Los sistemas han sido desarrollados para evitar cometer errores, logrando resarcir esta clase de errores. Además,  sería innecesario invertir presupuesto en personal para la recepción de papeles y se evita la pérdida de tiempo para los aspirantes. En el artículo periodístico [9] se indica que los beneficios de automatizar los procesos de una empresa son: un ahorro notable al momento de utilizar recursos, una mejor inversión del tiempo, se evitan pérdidas información a causa de errores, flujo de datos mucho más eficiente y obtención de  mejores condiciones frente a la competencia.
    Pero mediante el portal de internet de INFOTECHNOLOGY [10], el costo de un software a la medida es alto ronda desde los $500,000.00 hasta $1,000,000. o más. Atendiendo estos precios la manera más viable para las instituciones es optar por algo más económico, aunque estás no se adapten totalmente a su proceso.  
    El sistema está enfocado para beneficiar a todas las instituciones automatizando este proceso de manera efectiva y segura y principalmente a aquellas en donde su proceso de admisión escolar difiere demasiado de otras, y necesitan uno económico y a la medida.
    Para hacer esto posible nos auxiliaremos de nuestros conocimientos de Base de Datos, programación en distintos lenguajes, uso de frameworks, metodologías, Ingeniería de Software, administración de tiempo y recursos, cotizar el precio un sistema, propuesta de soluciones a los problemas que se presenten, seguridad en los datos, entre otros. 
